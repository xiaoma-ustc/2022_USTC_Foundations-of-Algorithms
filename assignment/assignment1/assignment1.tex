\documentclass[12pt, a4paper, oneside]{ctexart}
\usepackage{amsmath, amsthm, amssymb, graphicx}
\title{assignment1}
\author{xiaoma}
\date{2022.09.10}
\begin{document}
\maketitle
\subsection*{4.5-1}
\begin{itemize}
    \item [a.] $\mathbf{T} (n)=2\mathbf{T} (n / 4)+1$\\
    递归式满足形如$\mathbf{T} (n)=a\mathbf{T} (n/b)+f(n)$的情况,
    已知$a=2\geqslant 1,f(n)=\sqrt{n} $,则
    $$n^{\log _{4}2}=\sqrt{n}$$
    当$\varepsilon = \frac{1}{2} $时,有
    $$f(n)=O (n^{\log _{4}2-\varepsilon })$$
    所以$\mathbf{T} (n)=\Theta (\sqrt{n})$
    \item [b.] $\mathbf{T} (n)=2\mathbf{T} (n/4)+\sqrt{n}$\\
    参数同上,
    $$n^{\log _{4}2}=\sqrt{n}$$
    当$k = 0$时,有
    $$f(n)=O (n^{\log _{4}2}\lg^{k}n)$$
    所以$\mathbf{T} (n)=\Theta (\sqrt{n}\lg n)$
    \item [c.] $\mathbf{T} (n)=2\mathbf{T} (n/4)+n$\\
    参数同上,
    $$n^{\log _{4}2}=\sqrt{n}$$
    当$\varepsilon = \frac{1}{2} $时,有
    $$f(n)=O (n^{\log _{4}2+\varepsilon })$$
    所以$\mathbf{T} (n)=\Theta (n)$
    \item [d.] $\mathbf{T} (n)=2\mathbf{T} (n/4)+n^{2}$\\
    参数同上,
    $$n^{\log _{4}2}=\sqrt{n}$$
    当$\varepsilon = \frac{3}{2} $时,有
    $$f(n)=O (n^{\log _{4}2+\varepsilon })$$
    所以$\mathbf{T} (n)=\Theta (n^{2})$\\
\subsection*{4.5-4}
    可以\\
    由$$\mathbf{T} (n)=4\mathbf{T} (n / 2)+n^{2}\lg n$$
    知$$n^{\log _{2}4}=n^{2}$$
    当$k=1$时,有
    $$f(n)=O (n^{\log _{2}4}\lg^{k}n)$$
    所以$\mathbf{T} (n)=\Theta (n^{2}lg^{2}n)$
\end{itemize}
\end{document}