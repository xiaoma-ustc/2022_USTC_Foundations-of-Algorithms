\documentclass[12pt, a4paper, oneside]{ctexart}
\usepackage{amsmath, amsthm, amssymb, bm, color, framed, graphicx, hyperref, mathrsfs}
\usepackage{enumerate}
\usepackage{epstopdf}
\usepackage{float}
\usepackage{framed}
\usepackage[ruled,vlined]{algorithm2e}
\title{\textbf{assignment3}}
\author{Xiaoma}
\date{\today}
\linespread{1.5}
\definecolor{shadecolor}{RGB}{230, 245, 255}
\newcounter{problemname}
\newenvironment{problem}{\begin{shaded}\stepcounter{problemname}\par\noindent\textbf{题目\arabic{problemname}. }}{\end{shaded}\par}
\newenvironment{solution}{\par\noindent\textbf{解答. }}{\par}
\newenvironment{note}{\par\noindent\textbf{题目\arabic{problemname}的注记. }}{\par}

\begin{document}

\maketitle

\begin{problem}
\end{problem}
\begin{solution}
使用桶排序,基于元素的位数对元素进行分组,每个分组的排序方法为基数排序。\\
\pagebreak
\begin{algorithm*}
    \caption{SORT}
    \label{alg:algorithm}
    \KwIn{The array : $A[0...A.length - 1]$; The decimal of elements : r}
    \KwOut{The array sorted : $C[0...C.length - 1]$;}
    \BlankLine
    Initialize $d[0...d.length - 1]$;\\
    $d_{max}$ = 0;\\
    \For(){i = 0; i < A.length; ++i}{
        $d[i]$ = -1;\\
        t = $A[i]$;\\
        \While(){t != 0}{
            ++$d[i]$;\\
            t = t / r;
        }
        \If(){$d[i]$ > $d_{max}$}{
            $d_{max}$ = $d[i]$;\\
        }
    }
    create a new array $B[0...d_{max}]$;\\
    \For(){i = 0; i < $d_{max}$; ++i}{
        make $B[i]$ an empty list; 
    }
    \For(){i = 0; i < A.length; ++i}{
        insert $A[i]$ into list $B[d[i]]$;
    }
    \For(){i = 0; i < $d_{max}$; ++i}{
        RADIX-SORT($B[i]$);
    }
    \BlankLine
    concatenate lists $B[0]...B[d_{max} - 1]$ together in $C$;
    \BlankLine
    \Return{$C$}
\end{algorithm*}

第i个操作的代价为
$$d_{i}\begin{cases}
    i \quad if \ i = 2^{k},k\in \mathbb{N} \\
    1 \quad else
\end{cases}$$
n个操作序列的总代价为
$$\sum_{i=1}^{n}a_{i}$$
\begin{enumerate}
    \item 聚合分析:\\
    $$\sum_{i=1}^{n}a_{i} \leq \sum_{i=1}^{\lceil \log n \rceil }2^{i} + n \leq 5n \in O(n)$$
    摊还代价为$O(1)$
    \item 核算法: \\
    假设每个操作的摊还代价为3
    \begin{enumerate}
        \item 第一个操作的代价为1,剩余信用为2
        \item 假设在前$2^{i}$次操作后,信用不为负值,那么在进行后续
        $2^{i+1} - 1$次操作时,每次操作的代价为1,在进行第$2^{i+1}$次操作时,
        信用值至少为$2^{i+1}+1$,代价为$2^{i+1}$,信用为1。
    \end{enumerate}
    摊还代价为$O(1)$
    \item 势能法:\\
    设$k$为满足$2^{k}\leq i$的最大整数,则势函数为
    $$\Phi(D_{i})=\begin{cases}
        k + 3 \quad i = 2^{k}\\
        \Phi(D_{2^{k}})+2(i-2^{k}) \quad else
    \end{cases}$$
    则
    $$\Phi(D_{i})-\Phi(D_{i-1})=\begin{cases}
        -2^{k}+3 \quad i=2^{k}\\
        2 \quad else
    \end{cases}$$
    所以
    $$\sum_{i=1}^{n}a_{i}=3n=O(n)$$
    摊还代价为$O(1)$
\end{enumerate}
\end{solution}
\begin{problem}
    
\end{problem}
\begin{solution}
    
\end{solution}
\end{document}