\documentclass[12pt, a4paper, oneside]{ctexart}
\usepackage{amsmath, amsthm, amssymb, bm, color, framed, graphicx, hyperref, mathrsfs}
\usepackage{enumerate}
\usepackage{epstopdf}
\usepackage{float}
\usepackage{framed}
\usepackage[ruled,vlined]{algorithm2e}
\title{\textbf{assignment8}}
\author{Xiaoma}
\date{\today}
\linespread{1.5}
\definecolor{shadecolor}{RGB}{230, 245, 255}
\newcounter{problemname}
\newenvironment{problem}{\begin{shaded}\stepcounter{problemname}\par\noindent\textbf{题目\arabic{problemname}. }}{\end{shaded}\par}
\newenvironment{solution}{\par\noindent\textbf{解答. }}{\par}
\newenvironment{note}{\par\noindent\textbf{题目\arabic{problemname}的注记. }}{\par}

\begin{document}

\maketitle

\begin{problem}
\end{problem}
\begin{solution}
    \begin{itemize}
        \item 若被删去的边在原最小生成树中:\\
        原最小生成树被分为两部分,遍历整个图,寻找令两部分重新连接的权重
        最小的边,则得到新图的最小生成树。
        \item 若被删去的边不在原最小生成树中:\\
        则原最小生成树即为新图的最小生成树。
    \end{itemize}
    对于第一种情况,算法的时间复杂度为$O(E)$,则可以在与E成
    正比的时间内找到新图的最小生成树。
\end{solution}

\begin{problem}
    
\end{problem}
\begin{solution}
    假设新增的边为$(u,v)$,从$u$开始沿着最小生成树的边遍历到$v$,那么可以得到一个环$(u,v,...,u)$,选择
    该环中权值最大的点$e_{max}$
    \begin{itemize}
        \item 若$e_{max}$不是新增边,则将$e_{max}$从最小生成树中删去,
    并将新边$(u,v)$加入合并入最小生成树,得到新图的最下生成树。
        \item 若$e_{max}$是新增边,原最小生成树为新图的最小生成树。
    \end{itemize}
    算法的时间复杂度为$O(V)$。
\end{solution}

\begin{problem}
    
\end{problem}
\begin{solution}
    设起点集为$S_{1}$,其他点集为$S_{2}$,新增一个节点$s'$,设$s'$和$S_{1}$中的点都有边相连,且边的
    权值都为0,以$s'$为起点,使用dijkstra算法求$s'$到每个点的最短路径,得到最短路径树$T$,去掉$T$中
    $s'$与$S_{1}$中各点的边,则得到最短路径森林。记录每个起点到任意点的距离,则可以得到起点到任意点的最短
    路径。
\end{solution}

\begin{problem}
    
\end{problem}
\begin{solution}
    与3同理,首先采用Dijkstra算法得到最短路径森林,时间复杂度为$O(E \log V)$,然后枚举每个起点到终点的距离得到最短路径,时间复杂度为$O(V)$,
    则总时间复杂度为$O(E \log V)$。
\end{solution}
\end{document}